\documentclass[12pt]{article}
\usepackage{times}
\usepackage{fullpage}
\usepackage{gantt}

\setlength{\parindent}{0in}
\setlength{\parskip}{0.1in}

\begin{document}

\begin{center}
{\large CS5811 Project Proposal:}\\
{\large Flight Delay Prediction based on Bayesian Belief Networks}

{\small Jordon Dornbos\\}
{\small Liang Yan\\}

{\small September 18, 2014\\}

\end{center}

\section{Project Overview}

% Provide a paragraph introducing your project. Give a brief
% description: A more detailed description will be in Section
% \ref{sec:problem-desc}. Tell what motivates you to solve this
% problem.  
% You should have all the sections listed in this document. You may
% add more sections or change the order of the sections.  
% Throughout the document, assume a reader who has a good computing
% background. For example, don't describe what a tree is, but describe
% what the tree you use will be like. 
In this project we are going to develop an iPhone app which could
predict a probability that a flight will be delayed or not in a
specific future time. Since we do not know how difficult it will be,
we first focus on the flight between Chicago and New York, and the
prediction time is a day. We will extent the limitation as development 
continues.  
 
We are motivated to solve this problem because flight delays are a
fairly common thing and there seems to be no good source for
predicting this. Flight delays are generally only announced about an
hour or so before the flight is set to takeoff and they can change
unreliably. Our solution can't guarantee 100\% accuracy for predicting
a delay, but it will attempt to better prepare passengers for any
possible delays. 

\subsection{Problem description}
\label{sec:problem-desc}
% Start with an informal description, then define the problem as
% formally as possible. Provide background information where
% needed. Give the algorithms and heuristics that will be used. Give
% an illustrative example . 

I believe that almost everyone has had a bad experience with a flight being 
delayed, especially being delayed for something important. According to the
U.S. Department of Transportation, over 20 percent of all flights
arrive late.\cite{4732894}  
It is a major problem for the airplane company and also for us, passengers. 

Most of delays are caused by three reasons. First, and most important, is 
weather. Half of delays are actually caused by
weather\cite{5655493}. Second, it could  
be some mechanical problem\cite{6891588}, which is mostly decided by the plane
features such as the make, model, age and so on. Third, it could be
caused by a scheduling problem. If the last flight is delayed, this flight 
could be delayed also. \cite{5}

\subsection{State of the Art}

% What is the state of the art in terms of solving the problem you
% described? What contributions do you expect to make with this
% project to solving this problem and to your learning? Tell what your
% objective is and how you will measure success. 

Dr. Rebollo and Balakrishnan\cite{5} proposed amodel uses Random Forest (RF)
algorithms, based on temporal and spatial delay, also the local
arrival or departure delay situation, they proposed this new
network delay prediction model. The predictive performance of the model is
evaluated using the 100 most delayed OD pairs in the NAS:
the results show that given a 2-hour prediction horizon, the
average test error across these 100 OD(origin-destination) pairs is 19\% when
classifying delays as above or below 60 min. 

Dr. Lu and his team predict flight delays by creating a decision tree
from a large database. \cite{4732894}This solution is good at making a
simple tree to follow in order to  make predictions, but a new tree needs to be
created once more data is collected. 
Using neural networks, however, the graph stays the same, the weights are just
learned over time. Our solution will be learning how each input
parameter affects the delay probability. This should therefor be more
accurate than creating a graph  
of the most likely outcome based on a series of tests, such as the decision tree 
provides. 

\subsection{AI techniques}

% Which AI techniques are you planning to use? How is this project
% relevant to the Advanced AI course? 

We plan to use neural network to compute the probability of a delay. The neural
network will have multiple inputs describing the the three common issues
mentioned above. This includes weather data, information about the plane itself,
and the previous flight as well. 

We will train the neural network by feeding it data on current flights,
and back-propagating the correct value once the flight is over. We will capture
data on flights every day and add it to our database. This database will be used
to continually train the neural network. Over time it will learn the importance 
of each input, and should result in very accurate predictions.

Keeping a record of past flight information in our database will also allow us
to try out different parameters for the neural network. We can modify parameters,
train the network, and keep the parameters that performed best.

\section{Tasks}

% Define the tasks that you will need to accomplish for your
% project. Remember to describe the experiments you will carry
% out. Give a time estimate and tell who will do each task. 
% Use a project description scheme such as a Gantt chart. However,
% don't spend time or money on project description software unless you
% already have one that you are familiar with. 
% Tell which programming language and environment will be used, and
% whether you will be using any existing software. Provide a
% literature review here or where needed. 

\subsection{Platform}
iPhone app, Objective-C\\
Database, MySQL\\
Model training, Mathlab or write by our own (Python or Objective-c).\\
Experiment, check the correctness of the prediction.\\

\subsection{Schedule}
Preparation: \\
    GUI design, Jordon. \\
    Data collection, Liang

Main: Both\\
    Model construction \\  
    we are still not sure to use MathLab to train model or write our
    own program to implement it.

    Model embedded: \\
    Let the model work well in the iPhone app.

Integrity test: \\
    HCI improvement: Jordon \\ 
    Function modification:  Liang 

Verification: Both\\
    Including result analysis and report writing. The experiment is do
    a prediction based on the history data and check the correctness.

\subsection{Arrangement}
  \begin{gantt}{7}{8}
    \begin{ganttitle}
    \numtitle{1}{1}{8}{1}
    \end{ganttitle}
    \ganttbar{Gui design}{0}{2}
    \ganttbar{Data collection}{0}{2}
    \ganttbarcon[color=cyan]{Model construction}{2}{3}
    \ganttbarcon{Model embbeded}{5}{1}
    \ganttbarcon{Integrity test}{6}{1}
%    \ganttmilestone[color=cyan]{Milestone with color!}{4}
    \ganttbarcon{Verification}{7}{1}
  \end{gantt}

%\section{References}
% Give a list of references. There is no upper or lower limit on the number of references but cite only relevant publications.
\bibliographystyle{plain}
\bibliography{proposal-assignment}

\end{document}

